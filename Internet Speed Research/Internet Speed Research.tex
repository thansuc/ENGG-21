\documentclass[conference]{IEEEtran}
\IEEEoverridecommandlockouts
\usepackage{amsmath,amssymb,amsfonts}
\usepackage{algorithmic}
\usepackage{graphicx}
\usepackage{textcomp}
\usepackage{xcolor}
\def\BibTeX{{\rm B\kern-.05em{\sc i\kern-.025em b}\kern-.08em
    T\kern-.1667em\lower.7ex\hbox{E}\kern-.125emX}}
\begin{document}

\title{Performance Metrics for Evaluating Internet Speed Test in Symmetric Broadband\\
}

\author{\IEEEauthorblockN{1\textsuperscript{rd} Nathan Joshua Sucgang}
\IEEEauthorblockA{\textit{Department of Electronics, Computer, and Communications Engineering} \\
\textit{Ateneo De Manila University}\\
Quezon City, Philippines \\
nathan.sucgang@student.ateneo.edu}
\and
\IEEEauthorblockN{2\textsuperscript{rd} Geoffrey Miguel Guy}
\IEEEauthorblockA{\textit{Department of Electronics, Computer, and Communications Engineering} \\
\textit{Ateneo De Manila University}\\
Quezon City, Philippines \\
geoffrey.guy@student.ateneo.edu}
}

\maketitle

\begin{abstract}
    The internet has become more than a routine of everyone's life; 
    it has become a necessity. This paper aims to use statistics to identify the different factors that affect many aspects of the internet such as its download speed, upload speed, and ping. 
    The research looks at various factors for the causes and changes in the internet's performance such as the location of the household receiving the signal for their internet. 
    With this research information, the researchers of the paper intend to determine which possible factors hamper the internet's performance metrics in order to leave the issue in the hands of organizations who have the capacity to combat these same hindrances. 
    This data could also further improve the internet performance for the population, consequently improving a conventional part of urban societal life.
\end{abstract}

\begin{IEEEkeywords}
ping, download speed, upload speed, internet
\end{IEEEkeywords}

\section{Random Variables and Sources of Data}
\begin{enumerate}
    \item[1.]
    \textbf{Location}:
    The location of the closest test server when the test was being conducted.

    \textbf{Source}: Data gathered from the Ookla Speedtest using Raspberry Pi 4. 
    \item[2.]    
    \textbf{Download Speed}: 
    How fast you can receive data from the internet to your device (measured in Mbps).

    \textbf{Source}: Data gathered from the Ookla Speedtest using Raspberry Pi 4. 
    \item[3.]
    \textbf{Upload Speed}:
    How fast you can send data from your device to the internet (measured in Mbps). 
    
    \textbf{Source}: Data gathered from the Ookla Speedtest using Raspberry Pi 4. 
    \item[4.]
    \textbf{Ping}: 
    A type of latency measurement that sends a small packet of data from your device to a server and back (measured in ms).

    \textbf{Source}: Data gathered from the Ookla Speedtest using Raspberry Pi 4. 
    \item[5.] 
    \textbf{Jitter}: 
    Variation in latency over time (measured in ms).

    \textbf{Source}: Data gathered from the Ookla Speedtest using Raspberry Pi 4. 
    \item[6.] 
    \textbf{Latency}: 
    Time it takes for data to travel from your device to a server and back (measured in ms).

    \textbf{Source}: Data gathered from the Ookla Speedtest using Raspberry Pi 4. 
\end{enumerate}

\section{Normality of Data}
\begin{enumerate}
    \item[1.]    
    \textbf{Download Speed}: 
    \begin{enumerate}
        \item \textbf{Test}: Kolmogorov-Smirnov Test
        \item \textbf{Result}: D = 1, p-value $<$ 2.2e-16 (not normal)
    \end{enumerate}

    \item[2.]
    \textbf{Upload Speed}:
    \begin{enumerate}
        \item \textbf{Test}: Kolmogorov-Smirnov Test
        \item \textbf{Result}: D = 1, p-value $<$ 2.2e-16 (not normal)
    \end{enumerate}

    \item[3.]
    \textbf{Ping}:
    \begin{enumerate}
        \item \textbf{Test}: Kolmogorov-Smirnov Test
        \item \textbf{Result}: D = 0.81376, p-value $<$ 2.2e-16 (not normal)
    \end{enumerate}

    \item[4.] 
    \textbf{Download Jitter}: 
    \begin{enumerate}
        \item \textbf{Test}: Kolmogorov-Smirnov Test
        \item \textbf{Result}: D = 0.60492, p-value $<$ 2.2e-16 (not normal)
    \end{enumerate}

    \item[5.] 
    \textbf{Upload Jitter}: 
    \begin{enumerate}
        \item \textbf{Test}: Kolmogorov-Smirnov Test
        \item \textbf{Result}: D = 0.78504, p-value $<$ 2.2e-16 (not normal)
    \end{enumerate}

    \item[6.] 
    \textbf{Ping Jitter}: 
    \begin{enumerate}
        \item \textbf{Test}: Kolmogorov-Smirnov Test
        \item \textbf{Result}: D = 0.50309, p-value $<$ 2.2e-16 (not normal)
    \end{enumerate}

    \item[7.]   
    \textbf{Download Average Latency}: 
    \begin{enumerate}
        \item \textbf{Test}: Kolmogorov-Smirnov Test
        \item \textbf{Result}: D = 0.87014, p-value $<$ 2.2e-16 (not normal)
    \end{enumerate}

    \item[8.]   
    \textbf{Upload Average Latency}: 
    \begin{enumerate}
        \item \textbf{Test}: Kolmogorov-Smirnov Test
        \item \textbf{Result}: D = 0.88351, p-value $<$ 2.2e-16 (not normal)
    \end{enumerate}
\end{enumerate}

\section{Methodology}
\begin{figure}[htbp]
\centerline{\includegraphics[width=8cm,height=6cm,keepaspectratio]{Figures/experimental_setup.png}}
\caption{A scenario in evaluating internet speed test.}
\label{fig1}
\end{figure}
As shown in Fig. 1. The Methodology involve collecting data from the Ookla Speedtest using Raspberry Pi 4.
The experiment was conducted from June 18 to June 30, 2024 amounting to 1116 data points. The dataset was cleaned by removing any test from Globe Telecom, any error that came with misconfigured network.
Statistical analyses, including box plot, map graph, regression analysis, time series graph, was used to examined the relationship between download speed, upload speed, ping, jitter, and latency.
MS Excel and R was used to analyze the data. The statistical analysis was done to determine the factors that affect the internet speed test.

\section{Literature Review (Condensed Form)}

The internet operates by sending network access from an Internet Service Provider (ISP) from their Optic Line Terminal (OLT), which sends information through a line to an Optic Network Terminal (ONT), which is then given to customers via signal. In Thailand, Boonsongsrikul et al. [3] presented data that clearly shows how the speed of an internet network could exceed its advertised internet speed plan. However, this research gave a warning about the Optical Time Domain Reflectometer (OPTDR), which was the device that was used in evaluating the internet performance metrics during the experiment, that it is not exactly a device used for evaluating internet performance everyday. Furthermore, Sharma et al. [9] showed that changing an ISP will not make much of a difference if a client uses bottleneck WiFi, which is a concept that says the speed of WiFi that a gadget receives is different from the internet's access connection.

On the other hand, Bebortta and Das [2] showed that individuals should pay attention to the type of internet that they are using such as: broadband, cellular data, and etc. For instance, in their methodology they used 4G-LTE, or cellular data internet, to evaluate that the best performing online meeting platform is Microsoft Teams. On the other hand, Deng et al. [4] examined the differences in broadband ISP's speed performance, then concluded that the ISP does not really affect the broadband's performance through statistical tests such as the spearman rho. Likewise, Saengsai et al. [8] evaluated the performance of the broadband internet around the provinces in Thailand with their very own broadband internet speed dashboard, which concluded that ISP's should improve their upload speed.

Individuals also use different software programs to test internet speed, which may influence the measurement of their internet's performance metrics. On their official website, Speedtest [6] mentioned that they use the TCP Test Components and HTTP Legacy Fallback Testing to evaluate internet speed. The information gathering methods from speedtest simply measure how a unit of data (ping) and a chunk of data (download and upload) move around a computer and the network that is connected to this same computer. Similarly, a Bauer et al. [1] stated that Ookla usually results in higher measured data rates compared to other internet speed testing softwares, while also discussing how Ookla gathers its data by filtering 10\% of their fastest data and 30\% of their slowest data.

Research data also encompassed the topic of people's understanding of the internet. A Hasan and Woo [10] indicated that older adults, particularly those in America, evaluate the quality of their internet by relying on the performance of the internet on their devices and their “tech savvy” relatives. This information suggests that some old adult Americans do not understand internet 'jargon.' However, Ikhsan et al. [5] showed that some people, whose mean monthly income is 4 to 15 million rupiahs, rely on network quality, customer service, information quality, security, and privacy for the evaluation of their internet. ISP's monitor this kind of information to maintain customer loyalty.

Futhermore, in the Philippines, the department of Information and Communications Technology (DICT) [7] released a proposal to provide free internet access for public spaces. This internet is meant to be provided to public universities, government offices, and some open areas. However, the minimum goal internet download speed in the proposal is only 2 mbps. This paper shows how the Philippines understands the importance of providing internet to the public and an internet's performance metrics to reach their aim to provide free accessible internet for the Filipino people.

\section{Hypothesis}
Hypothesis 1: The internet speed will not exceed the customer's plan.
\begin{itemize}
    \item Alternative Hypothesis: The internet speed will exceed the customer's plan.
    \item Research Gap: There is no set parameter for the customer to consistently and continuously monitor their internet to determine the instances whether the internet could exceed or fall short from the customer's plan or not. Additionally, there are no definite measures to determine whether the provider of the internet gives their customer the actual or precise speed for their agreed internet plan.
\end{itemize}

\section{Figures}
\begin{figure}[htbp]
    \centerline{\includegraphics[width=10cm,height=8cm,keepaspectratio]{Figures/Picture1.png}}
    \caption{Box Plot for Internet Speed.}
    \label{fig2}
\end{figure}

\begin{figure}[htbp]
    \centerline{\includegraphics[width=10cm,height=8cm,keepaspectratio]{Figures/Picture2.png}}
    \caption{Box Plot for Ping.}
    \label{fig3}
\end{figure}

\begin{figure}[htbp]
    \centerline{\includegraphics[width=10cm,height=8cm,keepaspectratio]{Figures/Picture3.png}}
    \caption{Box Plot for Download Jitter.}
    \label{fig4}
\end{figure}

\begin{figure}[htbp]
    \centerline{\includegraphics[width=10cm,height=8cm,keepaspectratio]{Figures/Picture4.png}}
    \caption{Box Plot for Upload Jitter.}
    \label{fig4}
\end{figure}

\begin{figure}[htbp]
    \centerline{\includegraphics[width=10cm,height=8cm,keepaspectratio]{Figures/Picture5.png}}
    \caption{Box Plot for Ping Jitter.}
    \label{fig5}
\end{figure}

\begin{figure}[htbp]
    \centerline{\includegraphics[width=10cm,height=8cm,keepaspectratio]{Figures/Picture6.png}}
    \caption{Box Plot for Upload Latency.}
    \label{fig6}
\end{figure}

\begin{figure}[htbp]
    \centerline{\includegraphics[width=10cm,height=8cm,keepaspectratio]{Figures/Picture7.png}}
    \caption{Box Plot for Download Latency.}
    \label{fig7}
\end{figure}

\begin{figure}[htbp]
    \centerline{\includegraphics[width=10cm,height=8cm,keepaspectratio]{Figures/Picture17.png}}
    \caption{Time Series for Internet Speed.}
    \label{fig8}
\end{figure}

\begin{figure}[htbp]
    \centerline{\includegraphics[width=10cm,height=8cm,keepaspectratio]{Figures/Picture18.png}}
    \caption{Time Series for Ping.}
    \label{fig9}
\end{figure}

\begin{figure}[htbp]
    \centerline{\includegraphics[width=10cm,height=8cm,keepaspectratio]{Figures/Picture19.png}}
    \caption{Time Series for Download Jitter.}
    \label{fig10}
\end{figure}

\begin{figure}[htbp]
    \centerline{\includegraphics[width=10cm,height=8cm,keepaspectratio]{Figures/Picture20.png}}
    \caption{Time Series for Upload Jitter.}
    \label{fig11}
\end{figure}

\begin{figure}[htbp]
    \centerline{\includegraphics[width=10cm,height=8cm,keepaspectratio]{Figures/Picture21.png}}
    \caption{Time Series for Download Jitter.}
    \label{fig12}
\end{figure}

\begin{figure}[htbp]
    \centerline{\includegraphics[width=10cm,height=8cm,keepaspectratio]{Figures/Picture22.png}}
    \caption{Time Series for Upload Latency.}
    \label{fig13}
\end{figure}

\begin{figure}[htbp]
    \centerline{\includegraphics[width=10cm,height=8cm,keepaspectratio]{Figures/Picture23.png}}
    \caption{Time Series for Download Latency.}
    \label{fig14}
\end{figure}

\begin{figure}[htbp]
    \centerline{\includegraphics[width=10cm,height=8cm,keepaspectratio]{Figures/Picture24.png}}
    \caption{Class Interval for Download Speed.}
    \label{fig15}
\end{figure}

\begin{figure}[htbp]
    \centerline{\includegraphics[width=10cm,height=8cm,keepaspectratio]{Figures/Picture25.png}}
    \caption{Class Interval for Upload Speed.}
    \label{fig16}
\end{figure}

\begin{figure}[htbp]
    \centerline{\includegraphics[width=10cm,height=8cm,keepaspectratio]{Figures/Picture26.png}}
    \caption{Class Interval for Ping.}
    \label{fig17}
\end{figure}

\begin{figure}[htbp]
    \centerline{\includegraphics[width=10cm,height=8cm,keepaspectratio]{Figures/Picture27.png}}
    \caption{Class Interval for Download Jitter.}
    \label{fig17}
\end{figure}

\begin{figure}[htbp]
    \centerline{\includegraphics[width=10cm,height=8cm,keepaspectratio]{Figures/Picture29.png}}
    \caption{Class Interval for Upload Jitter.}
    \label{fig17}
\end{figure}

\begin{figure}[htbp]
    \centerline{\includegraphics[width=10cm,height=8cm,keepaspectratio]{Figures/Picture30.png}}
    \caption{Class Interval for Ping Jitter.}
    \label{fig17}
\end{figure}

\begin{figure}[htbp]
    \centerline{\includegraphics[width=10cm,height=8cm,keepaspectratio]{Figures/Picture31.png}}
    \caption{Class Interval for Upload Latency.}
    \label{fig17}
\end{figure}

\begin{figure}[htbp]
    \centerline{\includegraphics[width=10cm,height=8cm,keepaspectratio]{Figures/Picture32.png}}
    \caption{Class Interval for Download Latency.}
    \label{fig17}
\end{figure}

\begin{thebibliography}{00}
\bibitem{b1} S. Bauer et al., “Understanding broadband speed measurements”, MIT. Accessed: June 30,2024. [Online]. Available: https://groups.csail.mit.edu/ana/Publications/Understanding\_broadband\_
speed\_measurements\_bauer\_clark\_lehr\_TPRC\_2010.pdf.
\bibitem{b2} S. Bebortta and S. K. Das, "Assessing the Impact of Network Performance on Popular E-Learning Applications," 2020 Sixth International Conference on e-Learning (econf), Sakheer, Bahrain, 2020, pp. 61-65, doi: 10.1109/econf51404.2020.9385497.
\bibitem{b3} A. Boonsongsrikul et al., "Performance Metrics and Strategy for Evaluating Internet Speed Tests in Fixed Broadband Networks," 2024 IEEE International Conference on Big Data and Smart Computing (BigComp), Bangkok, Thailand, 2024, pp. 424-428, doi: 10.1109/BigComp60711.2024.00093.
\bibitem{b4} X. Deng et al., “Comparing Broadband ISP Performance using Big Data from M-Lab”, arXiv, Jan. 24, 2021, doi: 10.48550/arXiv.2101.09795.
\bibitem{b5} R. B. Ikhsan et al., "Customer Loyalty Based On Internet Service Providers-Service Quality," 2022 6th International Conference on Informatics and Computational Sciences (ICICoS), Semarang, Indonesia, 2022, pp. 18-23, doi: 10.1109/ICICoS56336.2022.9930615.
\bibitem{b6} C. Overturf., “How does Speedtest measure my network speeds?” SPEEDTEST. Accessed: June 30, 2024. [Online]. Available: https://help.speedtest.net/hc/en-us/articles/360038679354-How-does-Speedtest-measure-my-network-speeds
\bibitem{b7} E. M. Rio, Jr.,  “rules and regulations implementing republic act (r.a.) no. 10929 known as the free internet access in public places act”. DICT. Accessed: June 30, 2024. [Online]. Available: https://dict.gov.ph/wp-content/uploads/2017/12/IRR-RA-10929-Version-8.pdf (accessed June 30, 2024).
\bibitem{b8} A. Saengsai et al., "Broadband Internet Speed Dashboard for Sustainable Service Improvement in Thailand," 2024 IEEE International Conference on Big Data and Smart Computing (BigComp), Bangkok, Thailand, 2024, pp. 418-423, doi: 10.1109/BigComp60711.2024.00092. 
\bibitem{b9} R. Sharma et al., “Measuring the Prevalence of WiFi Bottlenecks in Home Access Networks,” arXiv, Nov. 9 2023, doi: 10.48550/arxiv.2311.05499.
\bibitem{b10} S. Hasan and W. Woo, “What's my Daily Value? Interpretation of network performance metrics in broadband consumer labels”, Association for Computing Machinery, New York, NY, USA, 2023, pp. 8-24, doi: 10.1145/3609396.3610546

\end{thebibliography}
\end{document}
