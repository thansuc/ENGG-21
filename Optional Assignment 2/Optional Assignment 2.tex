\documentclass{article}
\usepackage{blindtext}
\usepackage{amsmath}
\usepackage{multicol}
\usepackage{graphicx}
\usepackage[margin=1in]{geometry}
\begin{document}

Nathan Joshua B. Sucgang

\today

\begin{enumerate}
    \item For each module, give a short summary of the topics you learned. (3-5 sentences, 1 point per module, total 6 points).
    
    In the first module, we learn about how engineers like myself utilize statistics in our future careers and gain insights into big data and data scientists.
    The second module give a review of the foundational concepts of probability sets, which I rarely considered before. 
    Moving to the third module, we familiarize myself with PDF and CDF, essential for data analysis, along with distributions not covered in high school. 
    In the fourth module, we studied statistical inference and learned techniques for calculating confidence intervals for means and sample standard deviations.
    The fifth and last module, we learned nonparametric and parametric tests, we study 2k factorial design and methods to determine data sufficiency and usability.

    \item For each module, explain why it's important or how it contributes to your future research/industry work. (2-3 sentences, 2 points per module, total 12 points).
    
    At the start of the course, we are greeted with different kind of data analysis that can be used in our future jobs.  
    As we know, we walk before we run so that we lost along the way given this statement we learned different probability that are useful for statistic due to having interconnected between the different subject. 
    In our final requirement, we get to present a term paper given all we have learned to have a real experience of how to present our data gathered to the panelist.


    \item Then end with the most interesting topic you've learned from this class and explain why you found it interesting. (3-5 sentences, 2 points).
    
    The most interesting topic I have done is to be able to present a term paper that exceed the requirement for that specific paper.
    When I stumbled analysing my data or how to show the data that our professor are always available for consultation.
    I learned to use LaTeX even though is not actually required but it is a great experience and using programming like R in an actual classroom is also a great experience.
\end{enumerate} 
\end{document}